The long-run fraction of time in each state is determined by solving the following system of equations
%$$ \pi_S(q_S) = \pi_L q_{LS} + \pi_H q_{HS}$$
$$ \pi_L q_L = \pi_S q_{SL} + \pi_H q_{HL}$$
$$ \pi_H q_H = \pi_S q_{SH} + \pi_L q_{LH}$$
$$ \pi_S + \pi_H +\pi_L = 1 .$$

This results in that the long-run fractions of time in the states susceptible, lightly infected and heavily infected are $\pi_S = 0.9233$, $\pi_{I_H} = 0.0185$ and $\pi_{I_L} = 0.0582$. 

On average an individual is infected (either lightly or heavily) 28 days a year, since the long-run fractions of time in the states lightly infected and heavily infected sums to $7.67\%$.


FORSLAG: The long-run mean fractions of time spent in each state can be determined by solving the following system of equations:
%$$ \pi_S(q_S) = \pi_L q_{LS} + \pi_H q_{HS}$$
$$ \pi_L q_L = \pi_S q_{SL} + \pi_H q_{HL}$$
$$ \pi_H q_H = \pi_S q_{SH} + \pi_L q_{LH}$$
$$ \pi_S + \pi_H +\pi_L = 1, $$
where $\pi_i$ is the lon-run mean fraction of time spent in state $i$. This results in 

$$\pi_S = 0.9233 \quad \pi_{I_H} = 0.0185 \quad \pi_{I_L} = 0.0582$$

This means that an individual on average is infected, either lightly or heavily, 28 days per year, as a the mean fraction of time spent in those states sums to $7.67\%$.



