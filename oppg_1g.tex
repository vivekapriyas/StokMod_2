We use Little's law and define $L$ as the hospitals capacity to handle individuals with complications from a cold, $\lambda_h$ as the mean number of individuals that requires hospitalization, and $W$ as the average treatment time required so that the average number of individuals does not exceed the capacity. 

To determine $\lambda_h$, one has to calculate the long-run fraction of time an individual is susceptible, the average number of susceptible individuals and the average number of individuals getting infected each day. By solving the following set of equations, 
$$ \pi_S \lambda = \pi_I \mu$$ 
$$ \pi_S + \pi_I = 1 $$
the long-run fraction of time an individual is susceptible is determined to be $\pi_S = \frac{\mu}{\lambda + \mu} = \frac{100}{107} = 0.935.$
%$$ \pi_I = \frac{\lambda}{\lambda+\mu} $$
The average number of susceptible individuals is then 
$$ 5.26  \pi_S = 5.26 \cdot \frac{100}{107} = 4.9159 \text{ million individuals.}$$
Using that the transition rate $ q_{SI} = \lambda = 0.01$, the average number of individuals getting infected is $49159$ individuals a day. This makes the arrivalrate of individuals that requires hospitalization $1\%$ of the average number of individuals getting infected that day, i.e. $\lambda_h = 491.6$ individuals. 

The average treatment time required so that the average number of individuals in the hospital does not exceed the capacity is then determined to be $4.068$ days, using Little's law $W = \lambda_h/L$ with $L = 2000$ and $\lambda_h = 491.59$. 



