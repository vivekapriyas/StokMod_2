We use Little's law, where $L$ is the average number of individuals with complications from a cold in the hospital, $\lambda_h$ is the rate of arrival of individuals that requires hospitalization, and $W$ is the average treatment time. We wish to determine $W$ such that $L$ does not exceed the capacity of the hospital.
To determine $W$, we first have to determine $\lambda_h$. Since all individuals are independent of each other, we can calculate the long-run mean fraction of time one individual is susceptible and multiply this with the population size to get the average number of individuals susceptible in a given day.

By solving the following set of equations,
$$ \pi_S \lambda = \pi_I \mu$$
$$ \pi_S + \pi_I = 1 ,$$ 
the long-run mean fraction of time an individual is susceptible is determined to be $\pi_S = \frac{\mu}{\lambda + \mu} = \frac{100}{107} = 0.935.$
The average number of susceptible individuals per day is then 
$$ 5.26  \pi_S = 5.26 \cdot \frac{100}{107} = 4.9159 \text{ million individuals.}$$
Using that the transition rate from susceptible to infected for one individual is $\lambda = 0.01$, the average number of individuals getting infected per day is $49159$. As there is a $1\%$ probability that an infection will require hospitalization, $\lambda_h = 491.6$ individuals per day. 

Using Little's law, with $W = \lambda_h/L$ with $L = 2000$ and $\lambda_h = 491.59$, the maximal average required treatment time to ensure the average number of individuals in the hospital does not exceed its capacity is then determined to be $4.068$ days.
